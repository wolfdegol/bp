%%=============================================================================
%% Requirements2
%%=============================================================================

\chapter{Vergelijkende studie van manieren om een 3D-ruimte te bewerken}%
\label{ch:requirements2}

\subsection{Inleiding}

\subsection{Requirements oplijsten}
\subsubsection{Bewerkingstools voor mesh}
•	Vertex bewerking
•	Edge bewerking
•	Sculpting
•	Procedural Editing
•	UV editing
•	Topology editing
•	Retopology
•	Weight Painting

\subsubsection{Bewerkingstools voor point clouds}
Puntenselectie zo wel enkel als in groep
Translatie, rotatie en schaling
Verwijderen
Clipping & cropping => Bepalen hoeveel van de point cloud er kan worden gezien
Smoothing & Filtering => verwijderen van uitspringers & ongewenste artifacten
Interpolatie => invullen van missende of onvolledige data (zie inpainting Nerf)
Kleur en intensiteit bewerking van punten
point cloud segmentation
Gaussian Inpainting

\subsection{MoSCoW}

\subsubsection{Must-have}
\subsubsection{Should-have}
\subsubsection{Nice-to-have}

\subsection{Long List}

% Dan zoek je zoveel mogelijk alternatieven die in aanmerking komen om gebruikt te worden, m.a.w. al diegenen die je kan vinden. Je noemt ze in deze long list (die soms kan bestaan uit tientallen alternatieven) bij naam ,met eventueel vermelding van een website en een beschrijving in één zin. Elk alternatief toets je af aan de re-quirements, voor zover dit al mogelijk is aan de hand van informatie die je op de website vindt of via andere bronnen. Zaken die je niet kan verifiëren laat je gewoon open om later na te kijken of misschien zelfs te negeren (als het bv. gaat om een onbelangrijke feature, of als verschillende andere must-haves niet voldaan zijn). Je sorteert de long list dan volgens het aantal voldane requirements, en maakt hiereen overzichtelijke tabel van. Hopelijk heb je een aantal alternatieven overgehouden die voldoen aan alle must-haves en zoveel mogelijk should-haves en nice-to-haves.

\subsection{Conclusie}

