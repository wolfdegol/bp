%%=============================================================================
%% Methodologie
%%=============================================================================

\chapter{\IfLanguageName{dutch}{Methodologie}{Methodology}}%
\label{ch:methodologie}

%% TODO: In dit hoofstuk geef je een korte toelichting over hoe je te werk bent
%% gegaan. Verdeel je onderzoek in grote fasen, en licht in elke fase toe wat
%% de doelstelling was, welke deliverables daar uit gekomen zijn, en welke
%% onderzoeksmethoden je daarbij toegepast hebt. Verantwoord waarom je
%% op deze manier te werk gegaan bent.
%% 
%% Voorbeelden van zulke fasen zijn: literatuurstudie, opstellen van een
%% requirements-analyse, opstellen long-list (bij vergelijkende studie),
%% selectie van geschikte tools (bij vergelijkende studie, "short-list"),
%% opzetten testopstelling/PoC, uitvoeren testen en verzamelen
%% van resultaten, analyse van resultaten, ...
%%  
%% !!!!! LET OP !!!!!
%%
%% Het is uitdrukkelijk NIET de bedoeling dat je het grootste deel van de corpus
%% van je bachelorproef in dit hoofstuk verwerkt! Dit hoofdstuk is eerder een
%% kort overzicht van je plan van aanpak.
%%
%% Maak voor elke fase (behalve het literatuuronderzoek) een NIEUW HOOFDSTUK aan
%% en geef het een gepaste titel.
intro methodologie TODO
\subsection{Literatuuronderzoek}
 De eerste fase omvat een literatuurstudie waar alles omtrent het genereren, bekijken en bewerken van 3D-modellen wordt beschreven. Op die manier word er een beter beeld gecreëerd van alle beschikbare tools en manieren van aanpak die er zijn. Daarna word er twee maal een lijst van requirements opgesteld 
-

\subsection{MoSCoW}
Een requirementanalyse onderzoekt wat er precies nodig is in een applicatie. Welke requirements zijn van essentieel belang, welke functies zijn niet essentieel maar zouden er wel in moeten zitte, welke functies zijn ook niet essentieel, maar wel fijn om er bij te hebben, en welke functionaliteiten willen we niet implementeren. 

Hier wordt in feite een MoSCoW-analyse beschreven. MoSCoW staat voor Must Have, Should Have, Could Have en Wont Have. De bedoeling is om de verzameling van functionele- en niet-funcionele requirements op te lijsten en onder te verdelen in deze categorieën. Ten slotte krijgen we een lijst van de meest geschikte requirements en kunnen we daar verder mee aan de slag. 

\subsection{Vergelijkende studie omzetten naar 3D }

In de eerste studie word er achterhaald welke manier van aanpak om een reële omgeving om te zetten in een virtuele omgeving, de meest geschikte is. Dit is namelijk een van de deelvragen die mee zal moeten antwoorden hoe er best een applicatie wordt ontwikkeld die een dynamische, aanpasbare 3D ruimte kan genereren. 

Aan de hand van de informatie die er verworven wordt in de literatuurstudie, word er een requirementanalyse gedaan om te antwoorden op deze deelvraag. Meer specifiek maken we gebruik van de MoSCoW-methode 
2oplijsten van requirements & kleine beschrijving
3in tabel met functioneel/niet-functioneel & moscow 

\subsection{Vergelijkende studie bewerken van 3D }
1intro, wat word er hier besproken? TODO
2oplijsten van requirements & kleine beschrijving
3in tabel met functioneel/niet-functioneel & moscow 


\subsection{Proof of concept}
intro poc TODO
oplijsten succescriteria

\subsection{Conclusie} \subsection{Proof of concept}