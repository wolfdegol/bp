%%=============================================================================
%% Inleiding
%%=============================================================================

\chapter{\IfLanguageName{dutch}{Inleiding}{Introduction}}%
\label{ch:inleiding}

\section{\IfLanguageName{dutch}{Probleemstelling}{Problem Statement}}%
\label{sec:probleemstelling}

%%Uit je probleemstelling moet duidelijk zijn dat je onderzoek een meerwaarde heeft voor een concrete doelgroep. De doelgroep moet goed gedefinieerd en afgelijnd zijn. Doelgroepen als ``bedrijven,'' ``KMO's'', systeembeheerders, enz.~zijn nog te vaag. Als je een lijstje kan maken van de personen/organisaties die een meerwaarde zullen vinden in deze bachelorproef (dit is eigenlijk je steekproefkader), dan is dat een indicatie dat de doelgroep goed gedefinieerd is. Dit kan een enkel bedrijf zijn of zelfs één persoon (je co-promotor/opdrachtgever).%%


In 2022 ontstond er vanuit de opleiding ergotherapie aan de HoGent een vraag over de mogelijkheid van een Virtual-Reality (VR) applicatie als educatieve tool. De applicatie moet zich richten op het begrijpen van de behoeften van patiënten met dementie. Specifiek werd gekeken naar de mogelijkheid om een virtuele representatie te maken van een slaapkamer van een patiënt met dementie. Hierbij is het doel om studenten te helpen begrijpen welke aanpassingen nodig zijn om een patiënt met dementie het leven makkelijker te maken.

De bachelorproef van \textcite{Raes2023} suggereerde voorzichtig dat een dergelijk VR-hulpmiddel daadwerkelijk een meerwaarde zou kunnen bieden bij het leren ontwerpen van een dementievriendelijke omgeving. Deze bevindingen benadrukten het belang van verdere ontwikkeling en implementatie van VR-technologie in het onderwijs en de praktijk van ergotherapie.
Vervolgens onderzocht \textcite{Neufkens2023} in zijn bachelorproef welke game-engine het meest geschikt zou zijn voor het bouwen van zo'n virtuele omgeving. Als proof-of-concept creëerde hij een virtuele ziekenhuiskamer in de game-engine Unity. In zijn conclusie merkte hij op dat door een gebrek aan expertise het niet mogelijk was om objecten toe te voegen of te verwijderen in zijn applicatie, wat aangeeft dat verdere ontwikkeling en verfijning nodig zijn om het volledige potentieel van de VR-applicatie te benutten.
Recente ontwikkelingen in computer visie, generatieve AI, objectdetectie en -classificatie doen de vraag rijzen of het mogelijk is om het proces van het bouwen van een dergelijke 3D-ruimte gedeeltelijk of volledig te automatiseren. Dit zou studenten en docenten ergotherapie in staat stellen om zelf een ziekenhuiskamer te ontwerpen en aan te passen zonder enige voorkennis van 3D-ontwerp of programmeren. Dit automatiseringsproces kan op zichzelf als educatief worden beschouwd en biedt ook kostenbesparingen, aangezien het inhuren van een professional voor het ontwerpen en produceren van zo'n 3D-ruimte een prijzige aangelegenheid kan zijn. Dit opent nieuwe mogelijkheden voor efficiënter en toegankelijker onderwijs in ergotherapie, waarbij studenten praktische ervaring kunnen opdoen met het ontwerpen van dementievriendelijke omgevingen door middel van geavanceerde technologieën.


\section{\IfLanguageName{dutch}{Onderzoeksvraag}{Research question}}%
\label{sec:onderzoeksvraag}

%%Wees zo concreet mogelijk bij het formuleren van je onderzoeksvraag. Een onderzoeksvraag is trouwens iets waar nog niemand op dit moment een antwoord heeft (voor zover je kan nagaan). Het opzoeken van bestaande informatie (bv. ``welke tools bestaan er voor deze toepassing?'') is dus geen onderzoeksvraag. Je kan de onderzoeksvraag verder specifiëren in deelvragen. Bv.~als je onderzoek gaat over performantiemetingen, dan
%%

Welke technologieën zijn het meest geschikt om zo'n dynamisch, aanpasbare 3D ruimte te genereren in functie van een educatief middel voor de opleiding ergotherapie?

\section{\IfLanguageName{dutch}{Deelvragen}{Research subquestions}}%
\label{sec:deelvragen}


\begin{itemize}
%   in
\item Wat is de meest geschikte aanpak om een reële omgeving om te zetten in een 3D-ruimte?
\item Wat is de meest geschikte aanpak om die 3D-ruimte te bewerken?
\end{itemize}




\section{\IfLanguageName{dutch}{Onderzoeksdoelstelling}{Research objective}}%
\label{sec:onderzoeksdoelstelling}

%%Wat is het beoogde resultaat van je bachelorproef? Wat zijn de criteria voor succes? Beschrijf die zo concreet mogelijk. Gaat het bv.\ om een proof-of-concept, een prototype, een verslag met aanbevelingen, een vergelijkende studie, enz.
%%

Het beoogde resultaat van de bachelorproef is het onderzoeken van de mogelijkheid van een Virtual-Reality (VR) applicatie als educatieve tool voor het begrijpen en ontwerpen van dementievriendelijke omgevingen, specifiek gericht op studenten en docenten ergotherapie.
De criteria voor succes omvatten het ontwikkelen van een proof-of-concept die een virtuele representatie bied van een slaapkamer van een patiënt met dementie, die vervolgens aangepast kan worden zonder voorkennis van 3D-ontwerp of programmeren.


\section{\IfLanguageName{dutch}{Opzet van deze bachelorproef}{Structure of this bachelor thesis}}%
\label{sec:opzet-bachelorproef}

% Het is gebruikelijk aan het einde van de inleiding een overzicht te
% geven van de opbouw van de rest van de tekst. Deze sectie bevat al een aanzet
% die je kan aanvullen/aanpassen in functie van je eigen tekst.

De rest van deze bachelorproef is als volgt opgebouwd:

In Hoofdstuk~\ref{ch:stand-van-zaken} wordt een overzicht gegeven van de stand van zaken binnen het onderzoeksdomein, op basis van een literatuurstudie.

In Hoofdstuk~\ref{ch:methodologie} wordt de methodologie toegelicht en worden de gebruikte onderzoekstechnieken besproken om een antwoord te kunnen formuleren op de onderzoeksvragen.


In Hoofdstuk~\ref{ch:requirements1} wordt er een vergelijkende studie gedaan om te beslissen welk framework het meest geschikt is om automatisch een 3D ruimte te creëren.

In Hoofdstuk~\ref{ch:requirements2} wordt er een vergelijkende studie gedaan om te beslissen welk framework het meest geschikt is om automatisch een 3D ruimte te creëren.


In Hoofdstuk~\ref{ch:methodologie} wordt de methodologie toegelicht en worden de gebruikte onderzoekstechnieken besproken om een antwoord te kunnen formuleren op de onderzoeksvragen.

% TODO: Vul hier aan voor je eigen hoofstukken, één of twee zinnen per hoofdstuk

In Hoofdstuk~\ref{ch:conclusie}, tenslotte, wordt de conclusie gegeven en een antwoord geformuleerd op de onderzoeksvragen. Daarbij wordt ook een aanzet gegeven voor toekomstig onderzoek binnen dit domein.