%%=============================================================================
%% Requirements 1
%%=============================================================================

\chapter{Vergelijkende studie genereren van de 3D-ruimte}%
\label{ch:requirements1}

\subsection{Inleiding}

In dit hoofdstuk word er onderzocht welke manier van aanpak om een reële omgeving om te zetten in een virtuele 3D-omgeving de meest geschikte is. Dit gebeurt door de requirements op te sommen en onderverdelen in functioneel/niet-functionele requirements en vervolgens te prioriteren door middel van een MoSCoW analyse.
Vervolgens worden verschillende mogelijkheden van frameworks opgesomd met daarbij hun voor- en nadelen.
Ten slotte word op basis van de vooraf opgestelde requirements de lijst met frameworks beperkt tot één geschikt framework.
Dit framework is dan één van de requirements voor de volgende vergelijkende studie. Die onderzoekt welke tools het meest geschikt zijn om de gegenereerde 3D ruimte te kunnen bewerken in real-time.

\subsection{Functionele Requirements}

Functionele requirements beschrijven de technische noden die het framework moet vervullen om aan de  behoeften van de gebruikers te voldoen.

 Voor deze bachelorproef toegepaste informatica zijn de volgende functionele requirements opgesteld. (TODO bron functionele requirements)

\subsubsection{Als gebruiker wil ik geluiden of muziek toevoegen}

Volgens de studie van \textcite{Raes2023} kan het toevoegen van geluid in een simulatie het inlevingsvermogen vergroten, mits het volume beperkt blijft en de geluidskeuze aangenaam is. Het kunnen toevoegen en aanpassen van geluid is daarom een nuttige functionaliteit.

\subsubsection{Als gebruiker kan ik de gegenereerde omgeving aanpassen}

Dit houdt in dat de omgeving flexibel moet zijn. In overleg met copromotor Jana Van Damme is afgesproken dat het verplaatsen of verwijderen van objecten mogelijk moet zijn. Het aanpassen van de omgeving zorgt voor een interactief aspect, wat de aandachtsspanne van leerlingen verhoogt en een didactische meerwaarde biedt.

\begin{itemize}
    \item Als gebruiker kan ik objecten verplaatsen
    \item Als gebruiker kan ik objecten toevoegen
    \item Als gebruiker kan ik objecten verwijderen
\end{itemize}

\subsubsection{Als gebruiker kan ik de belichting van de omgeving aanpassen}

De studie van \textcite{Raes2023} haalt ook aan dat het verlichten van de omgeving een positieve invloed kan hebben op mensen met dementie. Daarmee is het ook aangewezen dat de verlichting eventueel kan worden aangepast in het resultaat van het genereren van de 3D-omgeving. De interactiviteit geeft ook hier een didactische meerwaarde.

\subsubsection{Het framework is compatibel met de beschikbare hardware en software}

Belangrijk is ook dat de gekozen software of algoritmes kunnen functioneren op de aanwezige hardware van de HoGent. Er is enkel een VR-headset beschikbaar, de Meta Quest 2.

\subsection{Niet-functionele requirements}

Niet-functionele requirements beschrijven de kwaliteitseisen en beperkingen van een systeem, die niet direct gerelateerd zijn aan specifieke functies of features. Deze eisen richten zich op hoe het systeem moet presteren en functioneren onder bepaalde omstandigheden.

In het kader van deze bachelorproef toegepaste informatica, zijn de volgende niet-functionele requirements van belang

\subsubsection{Gebruiksvriendelijk}

Het moet gemakkelijk zijn om de 3D-ruimte te creëren met deze manier van werken. Dit betekent dat de methode ook toegankelijk moet zijn voor mensen zonder modelleer- of programmeerkennis, zodat een brede gebruikersgroep ermee kan werken. Het framework aan zich kan geïmplementeerd worden binnen een applicatie waardoor de technische kant verborgen wordt voor de gebruiker, maar hoe minder tussenstappen er genomen moeten worden tussen het scannen van de omgeving tot het genereren van de 3D-ruimte, hoe beter.

\subsubsection{Haalbaarheid}

De manier van genereren moet haalbaar zijn voor een bachelorproef toegepaste informatica. Dit betekent dat de implementatie niet te complex mag zijn of te lang mag duren om te ontwikkelen, zodat het project binnen de gestelde tijdslijnen en middelen afgerond kan worden.


\subsubsection{Kost}

In samenspraak met de co-promotor van deze proef is vastgesteld dat er geen budget beschikbaar is. Dit houdt in dat er geen financiële middelen zijn voor abonnementen, het huren van rekenkracht of de aankoop van licenties, wat de keuze van technologieën en middelen beperkt.

\subsubsection{Realisme}

Een simulatie die dicht bij de werkelijkheid aanleunt, is in vele gevallen optimaal. Dit helpt gebruikers zich beter voor te bereiden op situaties, zich beter in te leven en uiteindelijk een aangenamere ervaring te hebben tijdens het gebruik van een realistische applicatie. De mate van realisme van de gebruikte tool is hierbij een cruciale factor.

\subsubsection{Betrouwbaarheid}

De aanpak moet consistent en voorspelbaar zijn, wat betekent dat dezelfde werkwijze steeds dezelfde resultaten moet opleveren met ten minste gemiddelde kwaliteit. Dit zorgt ervoor dat het systeem betrouwbaar is en er als ontwikkelaar en gebruiker op kan vertrouwd worden.

\subsection{MoSCoW}



\subsubsection{Must-have}
\subsubsection{Should-have}
\subsubsection{Nice-to-have}

\subsection{Long List}

% Dan zoek je zoveel mogelijk alternatieven die in aanmerking komen om gebruikt te worden, m.a.w. al diegenen die je kan vinden. Je noemt ze in deze long list (die soms kan bestaan uit tientallen alternatieven) bij naam ,met eventueel vermelding van een website en een beschrijving in één zin. Elk alternatief toets je af aan de re-quirements, voor zover dit al mogelijk is aan de hand van informatie die je op de website vindt of via andere bronnen. Zaken die je niet kan verifiëren laat je gewoon open om later na te kijken of misschien zelfs te negeren (als het bv. gaat om een onbelangrijke feature, of als verschillende andere must-haves niet voldaan zijn). Je sorteert de long list dan volgens het aantal voldane requirements, en maakt hiereen overzichtelijke tabel van. Hopelijk heb je een aantal alternatieven overgehouden die voldoen aan alle must-haves en zoveel mogelijk should-haves en nice-to-haves.

\subsection{Conclusie}